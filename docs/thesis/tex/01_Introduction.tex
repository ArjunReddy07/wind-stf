\chapter{Introduction}
Phenomena presenting high socio-economical relevance which are governed by complex dependencies of both spatial and temporal nature are found in diverse domains such as epidemiology, criminology, transportation, climate science and astrophysics \cite{atluri2018survey}.
Indeed, the ability to describe a system's behavior is most valuable on instances downstream in the arrow of time: forecasting \cite{armstrong2001principles, gartner2017analytics}.
Accurate, scalable and feasible rule-based forecasting modeling, however, remains elusive in many cases.
Especially as ubiquitous and continuous monitoring data become available, data-driven approaches emerge as a promising alternative.

Conventional data-driven approaches alone, however, have often shown to add limited value in spatio-temporal forecasting \cite{makridakis2018statistical}.
A major reason for this limitation lies on the assumptions they rely upon being typically violated in spatio-temporal settings.
Seasonality and spatial independency underlie most of the approaches from time series analysis, while earlier machine learning methods assume data instances are independent and identically distributed (i.i.d.) \cite{atluri2018survey}.
Recently, deep learning-based approaches have shown to be able to overcome this essentially by (a) modelling both spatial and temporal dependencies and (b) considering spatial similarities in terms less obvious than geographical proximity alone \cite{liu2017dcrnn, yao2018deep, li2019stgcn, zhang2019graphwavenet}.

In the context of renewables, accurately estimating power generation ahead of time poses a major obstacle in progressing towards carbon neutrality in power generation.
Heavily conditioned on weather and climate, harvesting energy from renewable sources is characterized by intermittency.
In the case of onshore wind power generation, climate change further aggravates this character, as wind speeds variability are expected to increase \cite{moemken2018future}.
Not accurately knowing how much wind power will be harvested in a certain time and region means power providers have to rely on unnecessarily large safety margins provided by conventional power plants for ensuring sufficient power supply.  
This ultimately hampers the expansion of wind farms and represents therefore a loss for the society, as part of the paid overall generated power is lost, as well as for the environment, as less environment-friendly power sources have to be relied upon \cite{delarue2015intermittency}.
For countries committed to large-scale initiatives such as the \textit{Energiewende} in Germany, this poses a major hindrance in decreasing overall carbon footprint in a sustainable fashion.
Accuracy on wind power generation forecasting hence has significant impact on both socio-economical and environmental aspects, in both short and long terms.

For a given installed capacity, wind power generation depends primarily on local wind speeds, which heavily vary in both time and space.
While the power generation can be predicted for each single region independently using historical data, we hypothesize that a significant increase in forecasting accuracy might be achieved by also considering inter-regional spatial dependencies. 

The objective of this work is twofold.
First, we delineate the state-of-the-art approaches for spatio-temporal forecasting in different domains.
Second, we apply selected approaches for forecasting wind power generation at the district-level in Germany.
By benchmarking against conventional approaches and single-regions forecasting horizons, we investigate whether more sophisticated modelling approaches add significant value in terms of accuracy in the use case of onshore wind power generation.  