\chapter{Experiments Settings}

Currently, all forecasting approaches are evaluated on a reduced case consisting of (a) 5 districts on northern Germany (DEF0C, DEF07, DEF0B, DEF05, DEF0E) located within 80 km distance from one another, (b) model inference time window from 2013-01-01 to 2015-06-22 and test time window from 2015-06-23 to 2015-06-29. We chose the test time window to be in a year period known to be less susceptible to wind gusts and other weather anomalies. Both models were evaluated in terms of predictions in capacity factors, with cross-districts uniform average of metrics. With regards of model tuning, only manual procedure has been carried out.

\vspace{1em}
\noindent
\textbf{HW-ES.} For preprocessing, relies on quantile transformation into a normal distribution, followed by an offsetting by the absolute value of the minimum of every scaled time series. The latter step is performed to ensure model inputs are strictly positive so as to allow for multiplicative seasonal approach in the HW-ES method. As for hyperparameters, we use additive trend, multiplicative seasonal, seasonal period of 7 steps (days).

\vspace{1em}
\noindent
\textbf{GWNet.} For preprocessing, relies on a Z-standard scaling. As for hyperparameters, we define most importantly the number of nodes (5), the sequence length (12), the learning rate (1E-3), and the learning decay rate (0.97).
