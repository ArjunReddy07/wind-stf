\resumo{%
  Predizer o comportamento de sistemas regidos por correlações temporais e espaciais é uma tarefa a que se tem atribuída crescente importância em diversas áreas de aplicação, desde neurociência, epidemiologia e criminologia a logística e transporte.
  Neste trabalho, delineamos o estado da arte para métodos de predição espaço-temporal e implementamos uma seleção desses métodos para a predição de geração de energia eólica no nível distrital na Alemanha.
  Na análise, levamos em conta tanto séries temporais com resolução horária entre 2000 e 2015, como também especificações de projeto e de instalação de turbinas eólicas individuais.
  Os modelos são avaliados em períodos não modelados e comparados com métodos estatísticos de previsão.
\\[3\baselineskip]
%
\textbf{Palavras-Chave}: Análise de Séries Temporais, Predição Espaço-Temporal, Aprendizagem de Máquina, Redes Neurais, Energias Renováveis, Energia Eólica.
}