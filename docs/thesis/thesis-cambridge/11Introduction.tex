\chapter{Introduction}
Phenomena presenting high socio-economical relevance which are governed by complex dependencies of both spatial and temporal nature are found in diverse domains such as epidemiology, criminology, transportation, climate science and astrophysics \cite{atluri2018datamining}.
Indeed, the ability to describe a system's behavior is most valuable on instances downstream in the arrow of time: forecasting \cite{armstrong2002principles}.
Accurate, scalable and feasible rule-based forecasting modeling, however, remains elusive in many cases.
Especially as ubiquitous and continuous monitoring data become available, data-driven approaches emerge as a promising alternative.

Conventional data-driven approaches alone, however, have often shown to add limited value in spatio-temporal forecasting \cite{makridakis2018waysforward}.
A major reason for this limitation lies on the assumptions they rely upon being typically violated in spatio-temporal settings.
Stationarity assumption most of the statistical approaches from time series analysis, while earlier machine learning methods assume data instances are independent and identically distributed (i.i.d.) \cite{atluri2018datamining}.
Recently, deep learning-based approaches have shown to be able to overcome this essentially by (a) modelling both spatial and temporal dependencies and (b) considering spatial similarities in terms less obvious than geographical proximity alone \cite{li2018dcrnn, liu2019st-mgcn, wu2019graphwavenet}.

In the context of renewables, accurately estimating power generation ahead of time poses a major obstacle in progressing towards carbon neutrality in power generation.
Heavily conditioned on weather and climate, harvesting energy from renewable sources is characterized by intermittency.
Wind power generation, for instance, depends primarily on local wind speeds, which heavily vary in both time and space.
Climate changes further aggravates this character, as wind speeds variability are expected to increase \cite{moemken2018windspeedchanges}.
Not accurately knowing how much wind power will be harvested in a certain time and region means power providers have to rely on unnecessarily larger safety margins provided by conventional power plants for ensuring sufficient power supply.
This ultimately hampers the expansion of wind farms and represents therefore a loss for the society, as part of the paid overall generated power is lost, as well as for the environment, as less environment-friendly power sources have to be relied upon \cite{delarue2015intermittency}.

For countries committed to large-scale initiatives such as the \textit{Energiewende} in Germany, this poses a major hindrance in decreasing overall carbon footprint in a sustainable fashion.
Accuracy on wind power generation forecasting hence has significant impact on both socio-economical and environmental aspects, in both short and long terms.

\section{Problem Statement}
In spatio-temporal problems, observations of a variable of interest over neighboring locations present not only temporal but also spatial dependencies.
While local time series can be predicted individually using conventional univariate statistical techniques, information contained in their spatial and spatio-temporal correlations represent a potential for improving forecasting accuracies. More sophisticated models that allow capturing these dependencies require however supporting evidence on their potential gains that justify the tipicaly longer development times they entail.

\section{Our Hypothesis}
We hypothesize that, in use cases dominated by spatio-temporal dependencies, significant forecasting performance gains can be achieved by spatio-temporal, multi-variate, Machine Learning-based approaches.

\section{Our Contribution}
First, we delineate the state-of-the-art approaches for temporal and spatio-temporal forecasting in different domains, including statistical, machine learning-based approaches.
Second, we apply selected approaches for forecasting weekly regional wind power generation in Germany.
By benchmarking against more conventional temporal, univariate, statistical approaches, we investigate to what extent more sophisticated modeling approaches add value in terms of accuracy in the use case of onshore wind power generation.